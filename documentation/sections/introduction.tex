\section{Предшествующие задачи}

\tab Решение задачи поиска оптимального размещения позволяет разрешать много прикладных проблем. Наиболее часто они возникают при планировании предприятий, работе с печатными платами, распространения инфраструктуры и других ресурсозатратных проектах.
\tab Рассмотрим некоторую хронологию возникновения задачи о размещении начиная с изложения Пьера де Ферма.
\subsection{Задача Штейнера о минимальном дереве}
\tab Фундаментальной основой стала проблема, сформулированная П.Фермом:
\begin{displayquote}
    \begin{center}
        Для заданных трех точек найти такую четвертую, что если из неё провести три отрезка в данные точки, то сумма этих трех отрезков даст наименьшую величину.
    \end{center}
\end{displayquote}
\tab Над получением решения трудились такие ученые как: Э. Торичелли, Б. Кавальери, Т. Симпсон, Ф. Хейнен, Ж. Бертран. В результате было получено геометрическое построение искомой точки, которую в последствии назвали точкой Ферма (иногда так же называют точкой Торричелли). 
\newline
\tab В 1934 году В. Ярник и О. Кесслер сформулировали обобщение задачи Ферма. Они сменили ограничение в три точки на произвольное конечное число. Поиск единственной точки в таком обобщении не смог получить достаточного внимания поэтому теперь их задача состояла в описании саязанных плоских графов наименьшей длинны, проходящих через данное конечное множество точек плоскости.  \cite{courant1941mathematics}
\newline
\tab Сейчас и задачу П. Ферма и задачу Ярника-Кесслера принято называть проблемой Штейнера. 

\subsection{Задача Притяжения – Отталкивания}
\tab Альтернативное развитие задача П. Ферма получила в обобщении немецкого экономиста Альфреда Вебера, который интерпретировал задачу как перевозку груза и назначил цену за еденицу расстояния для каждой точки, тем самым получив взвещенную задачу.
\newline
\tab В свою очередь задача Вебера-Ферма обобщается задачей притяжения - отталкивания, которая допускает отрицательные цены, тем самым делая для некоторых точек большие расстояния предпочтительнее. 
\newline
\tab Задача получила много вариантов решения для случая из трех точек. Они основываются на методах поиска углов и на их основе построения оптимального решения. Однако для случаев с большей размерностью эти методы бессильны. 
\newline
\tab Кун и Куэн в 1962 году предложили иттерационный алгоритм для задачи Ферма-Вебера. Он основывался на пошаговой минимизации суммы расстояний
\newline
\tab  Для задачи притяжения — отталкивания можно обратиться к помощи алгоритма, который предложили Чен, Хансен, Жомар и Туй в 1992.  
\newpage